\documentclass{article}
\usepackage[utf8]{inputenc}
\usepackage{amsmath, amssymb}
\usepackage{hyperref}
\usepackage{graphicx}
\usepackage{listings}
\usepackage{enumitem}

\title{Tycho Solver Documentation}
\author{Project Tycho}
\date{\today}

\begin{document}

\maketitle

\tableofcontents

\section{Overview}
The Tycho Solver is a Rust-based component designed to process real-time blockchain data and identify arbitrage opportunities using advanced graph algorithms. It receives block updates from a data feed, maintains an up-to-date graph of token exchanges, and applies a modified Bellman-Ford algorithm to detect negative cycles, which correspond to profitable arbitrage paths.

\section{System Architecture}
\begin{itemize}[leftmargin=*, label=--]
    \item \textbf{Tycho Feed:} An asynchronous task receives block updates and token data from an external source, forwarding them to the solver.
    \item \textbf{Solver:} Maintains the exchange graph, updates it with new data, and runs the arbitrage detection algorithm.
    \item \textbf{Channels:} Uses asynchronous channels (\texttt{tokio::sync::mpsc}) to communicate between the feed and the solver.
\end{itemize}

\section{Solver Workflow}
\begin{enumerate}[leftmargin=*, label=\arabic*.]
    \item \textbf{Initialization:} The solver initializes its graph structures and awaits block updates.
    \item \textbf{Block Update Handling:} On receiving a new block, the solver updates the graph with the latest exchange rates and liquidity data.
    \item \textbf{Arbitrage Detection:} The solver runs the modified Bellman-Ford algorithm to find negative cycles, which indicate arbitrage opportunities.
    \item \textbf{Result Export:} Detected opportunities are processed and can be exported or logged for further action.
\end{enumerate}

\section{Modified Bellman-Ford Algorithm}
\subsection{Purpose}
The Bellman-Ford algorithm is used to find the shortest paths in a weighted graph. In the context of the Tycho Solver, it is adapted to detect negative cycles, which correspond to arbitrage opportunities in the exchange graph.

\subsection{Data Structures}
\begin{itemize}[leftmargin=*, label=--]
    \item \texttt{Graph}: A directed graph where nodes represent tokens and edges represent exchange rates.
    \item \texttt{Edge Weights}: Edge weights are computed as the negative logarithm of the exchange rate, so that a negative cycle indicates a profitable arbitrage loop.
    \item \texttt{BigUint}: Used for precise, overflow-safe arithmetic on large integer values representing token amounts.
    \item \texttt{Rc<Vec<EdgeIndex>>}: To avoid unnecessary data copying, cycles found by the algorithm are stored as reference-counted vectors, reducing memory overhead and improving performance.
\end{itemize}

\subsection{Algorithm Steps}
\begin{enumerate}[leftmargin=*, label=\arabic*.]
    \item Initialize distances from the source node to all other nodes as infinity, except the source itself (zero).
    \item For each edge, relax the edge by updating the destination node's distance if a shorter path is found.
    \item Repeat the relaxation step for $|V|-1$ iterations, where $|V|$ is the number of vertices.
    \item After the main loop, check for negative cycles by verifying if any edge can still be relaxed. If so, reconstruct the cycle.
    \item Store detected cycles as \texttt{Rc<Vec<EdgeIndex>>} to minimize cloning and memory usage.
\end{enumerate}

\subsection{Optimizations}
\begin{itemize}[leftmargin=*, label=--]
    \item \textbf{Reference Counting:} By using \texttt{Rc<Vec<EdgeIndex>>}, the algorithm avoids deep cloning of cycle paths, which is especially beneficial when many cycles are detected.
    \item \textbf{Efficient Graph Updates:} Only relevant parts of the graph are updated on each block, minimizing recomputation.
    \item \textbf{BigUint Arithmetic:} Ensures that all calculations are safe from overflow, which is critical for financial computations.
\end{itemize}

\section{Example: Negative Cycle Detection}
\begin{lstlisting}[language=Rust, caption=Finding Negative Cycles]
// Pseudocode for negative cycle detection
for each node in graph {
    run bellman_ford from node
    if negative cycle found {
        store Rc<Vec<EdgeIndex>> for the cycle
    }
}
\end{lstlisting}

\section{Algorithm Details}

\subsection{High-Level Solver Flow}
Below is a high-level flowchart of the solver's main loop:

\begin{verbatim}
%% Mermaid flowchart (render with mermaid CLI or online)
flowchart TD
    A["Start: Receive Block Update"] --> B["Update Graph with New Pools/States"]
    B --> C["For Each Component"]
    C --> D["Run Bellman-Ford from Start Node"]
    D --> E["Relax Edges for N Iterations"]
    E --> F["Detect Negative Cycles"]
    F --> G["For Each Cycle"]
    G --> H["Optimize Input Amount (Golden Section Search)"]
    H --> I{"Profit > 0?"}
    I -- "Yes" --> J["Log/Export Arbitrage"]
    I -- "No" --> K["Skip"]
    J --> L["Wait for Next Block"]
    K --> L
    L --> A
\end{verbatim}

\subsection{Bellman-Ford Relaxation and Negative Cycle Detection}

\begin{verbatim}
%% Mermaid flowchart (render with mermaid CLI or online)
flowchart TD
    subgraph BellmanFord["Bellman-Ford Relaxation Loop"]
        B1["Initialize distances: ∞, source=0"]
        B2["Repeat for |V|-1 iterations:"]
        B3["  For each edge (u,v):"]
        B4["    If d[v] > d[u] + w(u,v):"]
        B5["      d[v] = d[u] + w(u,v)"]
        B6["      pred[v] = u"]
        B2 --> B3 --> B4 --> B5 --> B6
    end
    B1 --> B2
    B6 --> C["Check for negative cycles"]
    C --> D{"Any edge can still be relaxed?"}
    D -- "Yes" --> E["Reconstruct cycle using pred[]"]
    D -- "No" --> F["No negative cycle"]
    E --> G["Store cycle as Rc<Vec<EdgeIndex>>"]
    F --> H["Done"]
    G --> H
\end{verbatim}

\subsection{Component Interaction Sequence}

\begin{verbatim}
%% Mermaid sequence diagram (render with mermaid CLI or online)
sequenceDiagram
    participant S as Solver
    participant G as Graph
    participant B as BellmanFord
    participant O as Optimizer
    S->>G: Update with new block
    S->>B: Run Bellman-Ford
    B->>G: Relax edges
    B->>B: Repeat for N iterations
    B->>G: Check for negative cycles
    B-->>S: Return cycles
    S->>O: For each cycle, optimize input amount
    O-->>S: Return best amount, profit
    S->>S: If profit > 0, log/export arbitrage
    S->>S: Wait for next block
\end{verbatim}

\subsection{Pseudo-code}

\textbf{Main Solver Loop:}
\begin{lstlisting}[language=Rust]
loop {
    receive block update;
    update graph with new pools/states;
    for each component in graph {
        run Bellman-Ford from start node;
        for each negative cycle found {
            optimize input amount (golden section search);
            if profit > 0 {
                log/export arbitrage;
            }
        }
    }
}
\end{lstlisting}

\textbf{Bellman-Ford Relaxation:}
\begin{lstlisting}[language=Rust]
for i in 1..=V-1 {
    for each edge (u, v) {
        if d[v] > d[u] + w(u, v) {
            d[v] = d[u] + w(u, v);
            pred[v] = u;
        }
    }
}
\end{lstlisting}

\textbf{Negative Cycle Detection:}
\begin{lstlisting}[language=Rust]
for each edge (u, v) {
    if d[v] > d[u] + w(u, v) {
        // Negative cycle found
        reconstruct cycle using pred[];
        store cycle as Rc<Vec<EdgeIndex>>;
    }
}
\end{lstlisting}

\textbf{Profit Optimization (Golden Section Search):}
\begin{lstlisting}[language=Rust]
fn optimize_cycle_gss(cycle, graph, source, min, max, tol, max_iter, gas_price) {
    // Search for input amount that maximizes profit
    let (best_x, best_profit) = golden_section_search(min, max, tol, max_iter, |x| {
        let profit = simulate_cycle_profit(cycle, x, graph, source, gas_price);
        return profit;
    });
    return (best_x, best_profit);
}
\end{lstlisting}

\textbf{Edge Cases and Optimizations:}
\begin{itemize}
    \item Cycles are only considered if they start and end at the designated start token (e.g., WETH).
    \item Reference-counted vectors (\texttt{Rc<Vec<EdgeIndex>>}) are used to avoid unnecessary cloning.
    \item The algorithm skips cycles that revisit nodes to avoid infinite loops.
    \item Profit calculation includes gas costs, and only cycles with net positive profit are exported.
    \item The golden section search adapts the input range if invalid points are encountered.
\end{itemize}

\section{Conclusion}
The Tycho Solver combines real-time data processing with efficient graph algorithms to detect arbitrage opportunities in decentralized exchanges. The use of optimized data structures, such as reference-counted vectors and big integer arithmetic, ensures both performance and correctness.

\end{document} 